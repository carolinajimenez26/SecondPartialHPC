
%% bare_adv.tex
%% V1.4b
%% 2015/08/26
%% by Michael Shell
%% See: 
%% http://www.michaelshell.org/
%% for current contact information.
%%
%% This is a skeleton file demonstrating the advanced use of IEEEtran.cls
%% (requires IEEEtran.cls version 1.8b or later) with an IEEE Computer
%% Society journal paper.
%%
%% Support sites:
%% http://www.michaelshell.org/tex/ieeetran/
%% http://www.ctan.org/pkg/ieeetran
%% and
%% http://www.ieee.org/

%%*************************************************************************
%% Legal Notice:
%% This code is offered as-is without any warranty either expressed or
%% implied; without even the implied warranty of MERCHANTABILITY or
%% FITNESS FOR A PARTICULAR PURPOSE! 
%% User assumes all risk.
%% In no event shall the IEEE or any contributor to this code be liable for
%% any damages or losses, including, but not limited to, incidental,
%% consequential, or any other damages, resulting from the use or misuse
%% of any information contained here.
%%
%% All comments are the opinions of their respective authors and are not
%% necessarily endorsed by the IEEE.
%%
%% This work is distributed under the LaTeX Project Public License (LPPL)
%% ( http://www.latex-project.org/ ) version 1.3, and may be freely used,
%% distributed and modified. A copy of the LPPL, version 1.3, is included
%% in the base LaTeX documentation of all distributions of LaTeX released
%% 2003/12/01 or later.
%% Retain all contribution notices and credits.
%% ** Modified files should be clearly indicated as such, including  **
%% ** renaming them and changing author support contact information. **
%%*************************************************************************


% *** Authors should verify (and, if needed, correct) their LaTeX system  ***
% *** with the testflow diagnostic prior to trusting their LaTeX platform ***
% *** with production work. The IEEE's font choices and paper sizes can   ***
% *** trigger bugs that do not appear when using other class files.       ***                          ***
% The testflow support page is at:
% http://www.michaelshell.org/tex/testflow/


% IEEEtran V1.7 and later provides for these CLASSINPUT macros to allow the
% user to reprogram some IEEEtran.cls defaults if needed. These settings
% override the internal defaults of IEEEtran.cls regardless of which class
% options are used. Do not use these unless you have good reason to do so as
% they can result in nonIEEE compliant documents. User beware. ;)
%
%\newcommand{\CLASSINPUTbaselinestretch}{1.0} % baselinestretch
%\newcommand{\CLASSINPUTinnersidemargin}{1in} % inner side margin
%\newcommand{\CLASSINPUToutersidemargin}{1in} % outer side margin
%\newcommand{\CLASSINPUTtoptextmargin}{1in}   % top text margin
%\newcommand{\CLASSINPUTbottomtextmargin}{1in}% bottom text margin




%
\documentclass[10pt,journal,compsoc]{IEEEtran}
% If IEEEtran.cls has not been installed into the LaTeX system files,
% manually specify the path to it like:
% \documentclass[10pt,journal,compsoc]{../sty/IEEEtran}


% For Computer Society journals, IEEEtran defaults to the use of 
% Palatino/Palladio as is done in IEEE Computer Society journals.
% To go back to Times Roman, you can use this code:
%\renewcommand{\rmdefault}{ptm}\selectfont





% Some very useful LaTeX packages include:
% (uncomment the ones you want to load)



% *** MISC UTILITY PACKAGES ***
%
%\usepackage{ifpdf}
% Heiko Oberdiek's ifpdf.sty is very useful if you need conditional
% compilation based on whether the output is pdf or dvi.
% usage:
% \ifpdf
%   % pdf code
% \else
%   % dvi code
% \fi
% The latest version of ifpdf.sty can be obtained from:
% http://www.ctan.org/pkg/ifpdf
% Also, note that IEEEtran.cls V1.7 and later provides a builtin
% \ifCLASSINFOpdf conditional that works the same way.
% When switching from latex to pdflatex and vice-versa, the compiler may
% have to be run twice to clear warning/error messages.






% *** CITATION PACKAGES ***
%
\ifCLASSOPTIONcompsoc
  % The IEEE Computer Society needs nocompress option
  % requires cite.sty v4.0 or later (November 2003)
  \usepackage[nocompress]{cite}
\else
  % normal IEEE
  \usepackage{cite}
\fi
% cite.sty was written by Donald Arseneau
% V1.6 and later of IEEEtran pre-defines the format of the cite.sty package
% \cite{} output to follow that of the IEEE. Loading the cite package will
% result in citation numbers being automatically sorted and properly
% "compressed/ranged". e.g., [1], [9], [2], [7], [5], [6] without using
% cite.sty will become [1], [2], [5]--[7], [9] using cite.sty. cite.sty's
% \cite will automatically add leading space, if needed. Use cite.sty's
% noadjust option (cite.sty V3.8 and later) if you want to turn this off
% such as if a citation ever needs to be enclosed in parenthesis.
% cite.sty is already installed on most LaTeX systems. Be sure and use
% version 5.0 (2009-03-20) and later if using hyperref.sty.
% The latest version can be obtained at:
% http://www.ctan.org/pkg/cite
% The documentation is contained in the cite.sty file itself.
%
% Note that some packages require special options to format as the Computer
% Society requires. In particular, Computer Society  papers do not use
% compressed citation ranges as is done in typical IEEE papers
% (e.g., [1]-[4]). Instead, they list every citation separately in order
% (e.g., [1], [2], [3], [4]). To get the latter we need to load the cite
% package with the nocompress option which is supported by cite.sty v4.0
% and later.





% *** GRAPHICS RELATED PACKAGES ***
%
\ifCLASSINFOpdf
  % \usepackage[pdftex]{graphicx}
  % declare the path(s) where your graphic files are
  % \graphicspath{{../pdf/}{../jpeg/}}
  % and their extensions so you won't have to specify these with
  % every instance of \includegraphics
  % \DeclareGraphicsExtensions{.pdf,.jpeg,.png}
\else
  % or other class option (dvipsone, dvipdf, if not using dvips). graphicx
  % will default to the driver specified in the system graphics.cfg if no
  % driver is specified.
  % \usepackage[dvips]{graphicx}
  % declare the path(s) where your graphic files are
  % \graphicspath{{../eps/}}
  % and their extensions so you won't have to specify these with
  % every instance of \includegraphics
  % \DeclareGraphicsExtensions{.eps}
\fi
% graphicx was written by David Carlisle and Sebastian Rahtz. It is
% required if you want graphics, photos, etc. graphicx.sty is already
% installed on most LaTeX systems. The latest version and documentation
% can be obtained at: 
% http://www.ctan.org/pkg/graphicx
% Another good source of documentation is "Using Imported Graphics in
% LaTeX2e" by Keith Reckdahl which can be found at:
% http://www.ctan.org/pkg/epslatex
%
% latex, and pdflatex in dvi mode, support graphics in encapsulated
% postscript (.eps) format. pdflatex in pdf mode supports graphics
% in .pdf, .jpeg, .png and .mps (metapost) formats. Users should ensure
% that all non-photo figures use a vector format (.eps, .pdf, .mps) and
% not a bitmapped formats (.jpeg, .png). The IEEE frowns on bitmapped formats
% which can result in "jaggedy"/blurry rendering of lines and letters as
% well as large increases in file sizes.
%
% You can find documentation about the pdfTeX application at:
% http://www.tug.org/applications/pdftex





% *** MATH PACKAGES ***
%
%\usepackage{amsmath}
% A popular package from the American Mathematical Society that provides
% many useful and powerful commands for dealing with mathematics.
%
% Note that the amsmath package sets \interdisplaylinepenalty to 10000
% thus preventing page breaks from occurring within multiline equations. Use:
%\interdisplaylinepenalty=2500
% after loading amsmath to restore such page breaks as IEEEtran.cls normally
% does. amsmath.sty is already installed on most LaTeX systems. The latest
% version and documentation can be obtained at:
% http://www.ctan.org/pkg/amsmath





% *** SPECIALIZED LIST PACKAGES ***
%\usepackage{acronym}
% acronym.sty was written by Tobias Oetiker. This package provides tools for
% managing documents with large numbers of acronyms. (You don't *have* to
% use this package - unless you have a lot of acronyms, you may feel that
% such package management of them is bit of an overkill.)
% Do note that the acronym environment (which lists acronyms) will have a
% problem when used under IEEEtran.cls because acronym.sty relies on the
% description list environment - which IEEEtran.cls has customized for
% producing IEEE style lists. A workaround is to declared the longest
% label width via the IEEEtran.cls \IEEEiedlistdecl global control:
%
% \renewcommand{\IEEEiedlistdecl}{\IEEEsetlabelwidth{SONET}}
% \begin{acronym}
%
% \end{acronym}
% \renewcommand{\IEEEiedlistdecl}{\relax}% remember to reset \IEEEiedlistdecl
%
% instead of using the acronym environment's optional argument.
% The latest version and documentation can be obtained at:
% http://www.ctan.org/pkg/acronym


%\usepackage{algorithmic}
% algorithmic.sty was written by Peter Williams and Rogerio Brito.
% This package provides an algorithmic environment fo describing algorithms.
% You can use the algorithmic environment in-text or within a figure
% environment to provide for a floating algorithm. Do NOT use the algorithm
% floating environment provided by algorithm.sty (by the same authors) or
% algorithm2e.sty (by Christophe Fiorio) as the IEEE does not use dedicated
% algorithm float types and packages that provide these will not provide
% correct IEEE style captions. The latest version and documentation of
% algorithmic.sty can be obtained at:
% http://www.ctan.org/pkg/algorithms
% Also of interest may be the (relatively newer and more customizable)
% algorithmicx.sty package by Szasz Janos:
% http://www.ctan.org/pkg/algorithmicx




% *** ALIGNMENT PACKAGES ***
%
%\usepackage{array}
% Frank Mittelbach's and David Carlisle's array.sty patches and improves
% the standard LaTeX2e array and tabular environments to provide better
% appearance and additional user controls. As the default LaTeX2e table
% generation code is lacking to the point of almost being broken with
% respect to the quality of the end results, all users are strongly
% advised to use an enhanced (at the very least that provided by array.sty)
% set of table tools. array.sty is already installed on most systems. The
% latest version and documentation can be obtained at:
% http://www.ctan.org/pkg/array


%\usepackage{mdwmath}
%\usepackage{mdwtab}
% Also highly recommended is Mark Wooding's extremely powerful MDW tools,
% especially mdwmath.sty and mdwtab.sty which are used to format equations
% and tables, respectively. The MDWtools set is already installed on most
% LaTeX systems. The lastest version and documentation is available at:
% http://www.ctan.org/pkg/mdwtools


% IEEEtran contains the IEEEeqnarray family of commands that can be used to
% generate multiline equations as well as matrices, tables, etc., of high
% quality.


%\usepackage{eqparbox}
% Also of notable interest is Scott Pakin's eqparbox package for creating
% (automatically sized) equal width boxes - aka "natural width parboxes".
% Available at:
% http://www.ctan.org/pkg/eqparbox




% *** SUBFIGURE PACKAGES ***
%\ifCLASSOPTIONcompsoc
%  \usepackage[caption=false,font=footnotesize,labelfont=sf,textfont=sf]{subfig}
%\else
%  \usepackage[caption=false,font=footnotesize]{subfig}
%\fi
% subfig.sty, written by Steven Douglas Cochran, is the modern replacement
% for subfigure.sty, the latter of which is no longer maintained and is
% incompatible with some LaTeX packages including fixltx2e. However,
% subfig.sty requires and automatically loads Axel Sommerfeldt's caption.sty
% which will override IEEEtran.cls' handling of captions and this will result
% in non-IEEE style figure/table captions. To prevent this problem, be sure
% and invoke subfig.sty's "caption=false" package option (available since
% subfig.sty version 1.3, 2005/06/28) as this is will preserve IEEEtran.cls
% handling of captions.
% Note that the Computer Society format requires a sans serif font rather
% than the serif font used in traditional IEEE formatting and thus the need
% to invoke different subfig.sty package options depending on whether
% compsoc mode has been enabled.
%
% The latest version and documentation of subfig.sty can be obtained at:
% http://www.ctan.org/pkg/subfig




% *** FLOAT PACKAGES ***
%
%\usepackage{fixltx2e}
% fixltx2e, the successor to the earlier fix2col.sty, was written by
% Frank Mittelbach and David Carlisle. This package corrects a few problems
% in the LaTeX2e kernel, the most notable of which is that in current
% LaTeX2e releases, the ordering of single and double column floats is not
% guaranteed to be preserved. Thus, an unpatched LaTeX2e can allow a
% single column figure to be placed prior to an earlier double column
% figure.
% Be aware that LaTeX2e kernels dated 2015 and later have fixltx2e.sty's
% corrections already built into the system in which case a warning will
% be issued if an attempt is made to load fixltx2e.sty as it is no longer
% needed.
% The latest version and documentation can be found at:
% http://www.ctan.org/pkg/fixltx2e


%\usepackage{stfloats}
% stfloats.sty was written by Sigitas Tolusis. This package gives LaTeX2e
% the ability to do double column floats at the bottom of the page as well
% as the top. (e.g., "\begin{figure*}[!b]" is not normally possible in
% LaTeX2e). It also provides a command:
%\fnbelowfloat
% to enable the placement of footnotes below bottom floats (the standard
% LaTeX2e kernel puts them above bottom floats). This is an invasive package
% which rewrites many portions of the LaTeX2e float routines. It may not work
% with other packages that modify the LaTeX2e float routines. The latest
% version and documentation can be obtained at:
% http://www.ctan.org/pkg/stfloats
% Do not use the stfloats baselinefloat ability as the IEEE does not allow
% \baselineskip to stretch. Authors submitting work to the IEEE should note
% that the IEEE rarely uses double column equations and that authors should try
% to avoid such use. Do not be tempted to use the cuted.sty or midfloat.sty
% packages (also by Sigitas Tolusis) as the IEEE does not format its papers in
% such ways.
% Do not attempt to use stfloats with fixltx2e as they are incompatible.
% Instead, use Morten Hogholm'a dblfloatfix which combines the features
% of both fixltx2e and stfloats:
%
% \usepackage{dblfloatfix}
% The latest version can be found at:
% http://www.ctan.org/pkg/dblfloatfix


%\ifCLASSOPTIONcaptionsoff
%  \usepackage[nomarkers]{endfloat}
% \let\MYoriglatexcaption\caption
% \renewcommand{\caption}[2][\relax]{\MYoriglatexcaption[#2]{#2}}
%\fi
% endfloat.sty was written by James Darrell McCauley, Jeff Goldberg and 
% Axel Sommerfeldt. This package may be useful when used in conjunction with 
% IEEEtran.cls'  captionsoff option. Some IEEE journals/societies require that
% submissions have lists of figures/tables at the end of the paper and that
% figures/tables without any captions are placed on a page by themselves at
% the end of the document. If needed, the draftcls IEEEtran class option or
% \CLASSINPUTbaselinestretch interface can be used to increase the line
% spacing as well. Be sure and use the nomarkers option of endfloat to
% prevent endfloat from "marking" where the figures would have been placed
% in the text. The two hack lines of code above are a slight modification of
% that suggested by in the endfloat docs (section 8.4.1) to ensure that
% the full captions always appear in the list of figures/tables - even if
% the user used the short optional argument of \caption[]{}.
% IEEE papers do not typically make use of \caption[]'s optional argument,
% so this should not be an issue. A similar trick can be used to disable
% captions of packages such as subfig.sty that lack options to turn off
% the subcaptions:
% For subfig.sty:
% \let\MYorigsubfloat\subfloat
% \renewcommand{\subfloat}[2][\relax]{\MYorigsubfloat[]{#2}}
% However, the above trick will not work if both optional arguments of
% the \subfloat command are used. Furthermore, there needs to be a
% description of each subfigure *somewhere* and endfloat does not add
% subfigure captions to its list of figures. Thus, the best approach is to
% avoid the use of subfigure captions (many IEEE journals avoid them anyway)
% and instead reference/explain all the subfigures within the main caption.
% The latest version of endfloat.sty and its documentation can obtained at:
% http://www.ctan.org/pkg/endfloat
%
% The IEEEtran \ifCLASSOPTIONcaptionsoff conditional can also be used
% later in the document, say, to conditionally put the References on a 
% page by themselves.





% *** PDF, URL AND HYPERLINK PACKAGES ***
%
%\usepackage{url}
% url.sty was written by Donald Arseneau. It provides better support for
% handling and breaking URLs. url.sty is already installed on most LaTeX
% systems. The latest version and documentation can be obtained at:
% http://www.ctan.org/pkg/url
% Basically, \url{my_url_here}.


% NOTE: PDF thumbnail features are not required in IEEE papers
%       and their use requires extra complexity and work.
%\ifCLASSINFOpdf
%  \usepackage[pdftex]{thumbpdf}
%\else
%  \usepackage[dvips]{thumbpdf}
%\fi
% thumbpdf.sty and its companion Perl utility were written by Heiko Oberdiek.
% It allows the user a way to produce PDF documents that contain fancy
% thumbnail images of each of the pages (which tools like acrobat reader can
% utilize). This is possible even when using dvi->ps->pdf workflow if the
% correct thumbpdf driver options are used. thumbpdf.sty incorporates the
% file containing the PDF thumbnail information (filename.tpm is used with
% dvips, filename.tpt is used with pdftex, where filename is the base name of
% your tex document) into the final ps or pdf output document. An external
% utility, the thumbpdf *Perl script* is needed to make these .tpm or .tpt
% thumbnail files from a .ps or .pdf version of the document (which obviously
% does not yet contain pdf thumbnails). Thus, one does a:
% 
% thumbpdf filename.pdf 
%
% to make a filename.tpt, and:
%
% thumbpdf --mode dvips filename.ps
%
% to make a filename.tpm which will then be loaded into the document by
% thumbpdf.sty the NEXT time the document is compiled (by pdflatex or
% latex->dvips->ps2pdf). Users must be careful to regenerate the .tpt and/or
% .tpm files if the main document changes and then to recompile the
% document to incorporate the revised thumbnails to ensure that thumbnails
% match the actual pages. It is easy to forget to do this!
% 
% Unix systems come with a Perl interpreter. However, MS Windows users
% will usually have to install a Perl interpreter so that the thumbpdf
% script can be run. The Ghostscript PS/PDF interpreter is also required.
% See the thumbpdf docs for details. The latest version and documentation
% can be obtained at.
% http://www.ctan.org/pkg/thumbpdf


% NOTE: PDF hyperlink and bookmark features are not required in IEEE
%       papers and their use requires extra complexity and work.
% *** IF USING HYPERREF BE SURE AND CHANGE THE EXAMPLE PDF ***
% *** TITLE/SUBJECT/AUTHOR/KEYWORDS INFO BELOW!!           ***
\newcommand\MYhyperrefoptions{bookmarks=true,bookmarksnumbered=true,
pdfpagemode={UseOutlines},plainpages=false,pdfpagelabels=true,
colorlinks=true,linkcolor={black},citecolor={black},urlcolor={black},
pdftitle={},%<!CHANGE!
pdfsubject={Typesetting},%<!CHANGE!
pdfauthor={Carolina Jiménez Gómez, German David Gómez, Jhon Edinson Acevedo},%<!CHANGE!
pdfkeywords={Sobel Filter, GPU, CPU, OpenCV, C++}}%<^!CHANGE!
%\ifCLASSINFOpdf
%\usepackage[\MYhyperrefoptions,pdftex]{hyperref}
%\else
%\usepackage[\MYhyperrefoptions,breaklinks=true,dvips]{hyperref}
%\usepackage{breakurl}
%\fi
% One significant drawback of using hyperref under DVI output is that the
% LaTeX compiler cannot break URLs across lines or pages as can be done
% under pdfLaTeX's PDF output via the hyperref pdftex driver. This is
% probably the single most important capability distinction between the
% DVI and PDF output. Perhaps surprisingly, all the other PDF features
% (PDF bookmarks, thumbnails, etc.) can be preserved in
% .tex->.dvi->.ps->.pdf workflow if the respective packages/scripts are
% loaded/invoked with the correct driver options (dvips, etc.). 
% As most IEEE papers use URLs sparingly (mainly in the references), this
% may not be as big an issue as with other publications.
%
% That said, Vilar Camara Neto created his breakurl.sty package which
% permits hyperref to easily break URLs even in dvi mode.
% Note that breakurl, unlike most other packages, must be loaded
% AFTER hyperref. The latest version of breakurl and its documentation can
% be obtained at:
% http://www.ctan.org/pkg/breakurl
% breakurl.sty is not for use under pdflatex pdf mode.
%
% The advanced features offer by hyperref.sty are not required for IEEE
% submission, so users should weigh these features against the added
% complexity of use.
% The package options above demonstrate how to enable PDF bookmarks
% (a type of table of contents viewable in Acrobat Reader) as well as
% PDF document information (title, subject, author and keywords) that is
% viewable in Acrobat reader's Document_Properties menu. PDF document
% information is also used extensively to automate the cataloging of PDF
% documents. The above set of options ensures that hyperlinks will not be
% colored in the text and thus will not be visible in the printed page,
% but will be active on "mouse over". USING COLORS OR OTHER HIGHLIGHTING
% OF HYPERLINKS CAN RESULT IN DOCUMENT REJECTION BY THE IEEE, especially if
% these appear on the "printed" page. IF IN DOUBT, ASK THE RELEVANT
% SUBMISSION EDITOR. You may need to add the option hypertexnames=false if
% you used duplicate equation numbers, etc., but this should not be needed
% in normal IEEE work.
% The latest version of hyperref and its documentation can be obtained at:
% http://www.ctan.org/pkg/hyperref


% *** Do not adjust lengths that control margins, column widths, etc. ***
% *** Do not use packages that alter fonts (such as pslatex).         ***
% There should be no need to do such things with IEEEtran.cls V1.6 and later.
% (Unless specifically asked to do so by the journal or conference you plan
% to submit to, of course. )


% correct bad hyphenation here
\hyphenation{op-tical net-works semi-conduc-tor}








% \usepackage[spanish,english]{babel}
\usepackage[utf8]{inputenc}
\usepackage{url}
\usepackage[caption=false]{subfig}
\usepackage{graphicx}
\usepackage{mathtools}
\usepackage[spanish,english]{babel}
\selectlanguage{spanish}










\begin{document}
%
% paper title
% Titles are generally capitalized except for words such as a, an, and, as,
% at, but, by, for, in, nor, of, on, or, the, to and up, which are usually
% not capitalized unless they are the first or last word of the title.
% Linebreaks \\ can be used within to get better formatting as desired.
% Do not put math or special symbols in the title.
\title{Paralelización Del Filtro De Sobel 
Para La Detección De Bordes De Una Imagen 
Mediante La Tecnología CUDA  
}
%
%
% author names and IEEE memberships
% note positions of commas and nonbreaking spaces ( ~ ) LaTeX will not break
% a structure at a ~ so this keeps an author's name from being broken across
% two lines.
% use \thanks{} to gain access to the first footnote area
% a separate \thanks must be used for each paragraph as LaTeX2e's \thanks
% was not built to handle multiple paragraphs
%
%
%\IEEEcompsocitemizethanks is a special \thanks that produces the bulleted
% lists the Computer Society journals use for "first footnote" author
% affiliations. Use \IEEEcompsocthanksitem which works much like \item
% for each affiliation group. When not in compsoc mode,
% \IEEEcompsocitemizethanks becomes like \thanks and
% \IEEEcompsocthanksitem becomes a line break with idention. This
% facilitates dual compilation, although admittedly the differences in the
% desired content of \author between the different types of papers makes a
% one-size-fits-all approach a daunting prospect. For instance, compsoc 
% journal papers have the author affiliations above the "Manuscript
% received ..."  text while in non-compsoc journals this is reversed. Sigh.

\author{Carolina~Jiménez~Gómez,
        German~David~Gómez,
        y~Jhon~Edinson~Acevedo% <-this % stops a space
% \IEEEcompsocitemizethanks{\IEEEcompsocthanksitem M. Shell was with the Department
% of Electrical and Computer Engineering, Georgia Institute of Technology, Atlanta,
% GA, 30332.\protect\\
% % note need leading \protect in front of \\ to get a newline within \thanks as
% % \\ is fragile and will error, could use \hfil\break instead.
% E-mail: see http://www.michaelshell.org/contact.html
% \IEEEcompsocthanksitem J. Doe and J. Doe are with Anonymous University.}% <-this % stops a space
% \thanks{Manuscript received April 19, 2005; revised August 26, 2015.}
}

% note the % following the last \IEEEmembership and also \thanks - 
% these prevent an unwanted space from occurring between the last author name
% and the end of the author line. i.e., if you had this:
% 
% \author{....lastname \thanks{...} \thanks{...} }
%                     ^------------^------------^----Do not want these spaces!
%
% a space would be appended to the last name and could cause every name on that
% line to be shifted left slightly. This is one of those "LaTeX things". For
% instance, "\textbf{A} \textbf{B}" will typeset as "A B" not "AB". To get
% "AB" then you have to do: "\textbf{A}\textbf{B}"
% \thanks is no different in this regard, so shield the last } of each \thanks
% that ends a line with a % and do not let a space in before the next \thanks.
% Spaces after \IEEEmembership other than the last one are OK (and needed) as
% you are supposed to have spaces between the names. For what it is worth,
% this is a minor point as most people would not even notice if the said evil
% space somehow managed to creep in.



% The paper headers
\markboth{Octubre 20 del 2017}%
{Shell \MakeLowercase{\textit{et al.}}: Bare Advanced Demo of IEEEtran.cls for IEEE Computer Society Journals}
% The only time the second header will appear is for the odd numbered pages
% after the title page when using the twoside option.
% 
% *** Note that you probably will NOT want to include the author's ***
% *** name in the headers of peer review papers.                   ***
% You can use \ifCLASSOPTIONpeerreview for conditional compilation here if
% you desire.



% The publisher's ID mark at the bottom of the page is less important with
% Computer Society journal papers as those publications place the marks
% outside of the main text columns and, therefore, unlike regular IEEE
% journals, the available text space is not reduced by their presence.
% If you want to put a publisher's ID mark on the page you can do it like
% this:
%\IEEEpubid{0000--0000/00\$00.00~\copyright~2015 IEEE}
% or like this to get the Computer Society new two part style.
%\IEEEpubid{\makebox[\columnwidth]{\hfill 0000--0000/00/\$00.00~\copyright~2015 IEEE}%
%\hspace{\columnsep}\makebox[\columnwidth]{Published by the IEEE Computer Society\hfill}}
% Remember, if you use this you must call \IEEEpubidadjcol in the second
% column for its text to clear the IEEEpubid mark (Computer Society journal
% papers don't need this extra clearance.)



% use for special paper notices
%\IEEEspecialpapernotice{(Invited Paper)}



% for Computer Society papers, we must declare the abstract and index terms
% PRIOR to the title within the \IEEEtitleabstractindextext IEEEtran
% command as these need to go into the title area created by \maketitle.
% As a general rule, do not put math, special symbols or citations
% in the abstract or keywords.
\IEEEtitleabstractindextext{%
\begin{abstract}
This article discusses the implementation of the Sobel filter with CUDA technology to take advantage of the Graphic Processing Unit (GPU), making a comparison between the different types of memory that the GPU has and it’s optimizations in different computational algorithms. We give a brief description of the algorithm of Sobel and an example of convolution on the images, to after deepen into the differences between different types of memory. We show comparison tables between the different algorithms and makes an analysis of the acceleration between them.

\end{abstract}

% Note that keywords are not normally used for peerreview papers.
\begin{IEEEkeywords}
CUDA, GPU, C++, OpenCV, CPU, Paralelismo, Optimización, Filtro Sobel, Procesamiento de imágenes.
\end{IEEEkeywords}}


% make the title area
\maketitle


% To allow for easy dual compilation without having to reenter the
% abstract/keywords data, the \IEEEtitleabstractindextext text will
% not be used in maketitle, but will appear (i.e., to be "transported")
% here as \IEEEdisplaynontitleabstractindextext when compsoc mode
% is not selected <OR> if conference mode is selected - because compsoc
% conference papers position the abstract like regular (non-compsoc)
% papers do!
\IEEEdisplaynontitleabstractindextext
% \IEEEdisplaynontitleabstractindextext has no effect when using
% compsoc under a non-conference mode.


% For peer review papers, you can put extra information on the cover
% page as needed:
% \ifCLASSOPTIONpeerreview
% \begin{center} \bfseries EDICS Category: 3-BBND \end{center}
% \fi
%
% For peerreview papers, this IEEEtran command inserts a page break and
% creates the second title. It will be ignored for other modes.
\IEEEpeerreviewmaketitle

\ifCLASSOPTIONcompsoc
\IEEEraisesectionheading{\section{Resumen}\label{sec:introduccion}}
\else
\section{Resumen}
\label{sec:resumen}
\fi
% Computer Society journal (but not conference!) papers do something unusual
% with the very first section heading (almost always called "Introduction").
% They place it ABOVE the main text! IEEEtran.cls does not automatically do
% this for you, but you can achieve this effect with the provided
% \IEEEraisesectionheading{} command. Note the need to keep any \label that
% is to refer to the section immediately after \section in the above as
% \IEEEraisesectionheading puts \section within a raised box.




% The very first letter is a 2 line initial drop letter followed
% by the rest of the first word in caps (small caps for compsoc).
% 
% form to use if the first word consists of a single letter:
% \IEEEPARstart{A}{demo} file is ....
% 
% form to use if you need the single drop letter followed by
% normal text (unknown if ever used by the IEEE):
% \IEEEPARstart{A}{}demo file is ....
% 
% Some journals put the first two words in caps:
% \IEEEPARstart{T}{his demo} file is ....
% 
% Here we have the typical use of a "T" for an initial drop letter
% and "HIS" in caps to complete the first word.
\IEEEPARstart{E}{n} este artículo se hablará sobre la implementación del filtro de Sobel con la tecnología CUDA para el aprovechamiento de la Unidad de Procesamiento Gráfico (GPU, por sus siglas en Inglés), haciendo una comparación entre los diferentes tipos de memoria que tiene la GPU y con las cuales se pueden hacer optimizaciones de diferentes algoritmos computacionales. Se da una breve descripción del algoritmo de Sobel y un ejemplo de convolución sobre las imágenes, para después adentrarnos en las diferencias sustanciales entre los diferentes tipos de memoria. Se muestran gráficas de comparación entre los diferentes algoritmos y se hace un análisis de aceleración entre ellos.


% \subsection{Subsection Heading Here}
% Subsection text here.

% needed in second column of first page if using \IEEEpubid
%\IEEEpubidadjcol

% \subsubsection{Subsubsection Heading Here}
% Subsubsection text here.


% An example of a floating figure using the graphicx package.
% Note that \label must occur AFTER (or within) \caption.
% For figures, \caption should occur after the \includegraphics.
% Note that IEEEtran v1.7 and later has special internal code that
% is designed to preserve the operation of \label within \caption
% even when the captionsoff option is in effect. However, because
% of issues like this, it may be the safest practice to put all your
% \label just after \caption rather than within \caption{}.
%
% Reminder: the "draftcls" or "draftclsnofoot", not "draft", class
% option should be used if it is desired that the figures are to be
% displayed while in draft mode.
%
%\begin{figure}[!t]
%\centering
%\includegraphics[width=2.5in]{myfigure}
% where an .eps filename suffix will be assumed under latex, 
% and a .pdf suffix will be assumed for pdflatex; or what has been declared
% via \DeclareGraphicsExtensions.
%\caption{Simulation results for the network.}
%\label{fig_sim}
%\end{figure}

% Note that the IEEE typically puts floats only at the top, even when this
% results in a large percentage of a column being occupied by floats.
% However, the Computer Society has been known to put floats at the bottom.


% An example of a double column floating figure using two subfigures.
% (The subfig.sty package must be loaded for this to work.)
% The subfigure \label commands are set within each subfloat command,
% and the \label for the overall figure must come after \caption.
% \hfil is used as a separator to get equal spacing.
% Watch out that the combined width of all the subfigures on a 
% line do not exceed the text width or a line break will occur.
%
%\begin{figure*}[!t]
%\centering
%\subfloat[Case I]{\includegraphics[width=2.5in]{box}%
%\label{fig_first_case}}
%\hfil
%\subfloat[Case II]{\includegraphics[width=2.5in]{box}%
%\label{fig_second_case}}
%\caption{Simulation results for the network.}
%\label{fig_sim}
%\end{figure*}
%
% Note that often IEEE papers with subfigures do not employ subfigure
% captions (using the optional argument to \subfloat[]), but instead will
% reference/describe all of them (a), (b), etc., within the main caption.
% Be aware that for subfig.sty to generate the (a), (b), etc., subfigure
% labels, the optional argument to \subfloat must be present. If a
% subcaption is not desired, just leave its contents blank,
% e.g., \subfloat[].


% An example of a floating table. Note that, for IEEE style tables, the
% \caption command should come BEFORE the table and, given that table
% captions serve much like titles, are usually capitalized except for words
% such as a, an, and, as, at, but, by, for, in, nor, of, on, or, the, to
% and up, which are usually not capitalized unless they are the first or
% last word of the caption. Table text will default to \footnotesize as
% the IEEE normally uses this smaller font for tables.
% The \label must come after \caption as always.
%
%\begin{table}[!t]
%% increase table row spacing, adjust to taste
%\renewcommand{\arraystretch}{1.3}
% if using array.sty, it might be a good idea to tweak the value of
% \extrarowheight as needed to properly center the text within the cells
%\caption{An Example of a Table}
%\label{table_example}
%\centering
%% Some packages, such as MDW tools, offer better commands for making tables
%% than the plain LaTeX2e tabular which is used here.
%\begin{tabular}{|c||c|}
%\hline
%One & Two\\
%\hline
%Three & Four\\
%\hline
%\end{tabular}
%\end{table}


% Note that the IEEE does not put floats in the very first column
% - or typically anywhere on the first page for that matter. Also,
% in-text middle ("here") positioning is typically not used, but it
% is allowed and encouraged for Computer Society conferences (but
% not Computer Society journals). Most IEEE journals/conferences use
% top floats exclusively. 
% Note that, LaTeX2e, unlike IEEE journals/conferences, places
% footnotes above bottom floats. This can be corrected via the
% \fnbelowfloat command of the stfloats package.


\section{Introducción}

El filtro de Sobel se utiliza en el procesamiento de imágenes y visión por computadora, particularmente dentro de los algoritmos de detección de bordes. Esto se logra gracias a la información proporcionada por las fronteras de los objetos que  aparecen en una imagen como las discontinuidades en los niveles de grises. En concreto el filtro de Sobel se trata de un filtro de aproximación al gradiente. El cálculo de la derivada direccional de una función nos habla de cómo se producen los cambios en tal dirección, tales cambios, que se asocian con las altas frecuencias, suelen corresponder a los bordes de los objetos presentes en las imágenes\cite{SD_openCV}.

Para este fin se debe procesar la imagen pixel a pixel, lo que se traduce en muchas operaciones secuenciales y por ende mucho tiempo de procesamiento, por lo que se plantea como solución a este problema la paralelización por medio de la tecnología CUDA\cite{CUDAZone} de Nvidia con la cual se espera obtener un rendimiento computacional mucho mayor.

El objetivo de este artículo es mostrar las ventajas de usar paralelización usando una GPU y el lenguaje CUDA de Nvidia para procesamiento de grandes cantidades de datos. En este caso los datos a procesar son imágenes, por esta razón, en este artículo no se pretende demostrar las bases matemáticas sobre las cuales funcionan la conversión a escala de grises, convolución y el filtro de Sobel, que son procesos aplicados a cada imagen.


\section{Materiales y métodos}

Los algoritmos fueron procesados en un computador Intel(R) Core(TM) i7-4790K CPU @ 4.00GHz de arquitectura x86\textunderscore64, con 8 CPUs,  L1d cache y L1i cache de 32K, L2 cache de 256K y L3 cache de 8192K; un Xeon E3-1200 v3/4th Gen Core Processor Integrated Graphics Controller de Intel Corporation.


Se utiliza la tecnología CUDA para la paralelización de los algoritmos, la librería de OpenCV\cite{opencv} para el tratamiento general de imágenes y para la implementación del algoritmo de forma secuencial; se utiliza el lenguaje de programación C++\cite{c++} para las implementaciones.

La librería OpenCV es una biblioteca libre de visión artificial originalmente desarrollada por Intel. OpenCV pretende proporcionar un entorno de desarrollo fácil de utilizar y altamente eficiente. Esto se ha logrado realizando su programación en código C y C++ optimizados, aprovechando las capacidades que proveen los procesadores multinúcleo\cite{4rios}.


Se utiliza OpenCV para la carga de las imágenes y la posterior escritura de la imagen resultante  en todos los algoritmos.

Para la realización de las pruebas experimentales se tuvieron en cuenta diez imágenes con tamaños entre 2K y 8K sobre las cuales cada algoritmo fue ejecutado veinte veces para posteriormente realizar un promedio de estos  valores y tener un tiempo de ejecución mucho más confiable.

Todos los algoritmos implementados pueden ser encontrados en \cite{repo}


\subsection{Tecnología CUDA}

La tecnología CUDA es una arquitectura de computación paralela desarrollada por uno de
los mayores fabricantes de tarjetas gráficas del mercado, NVIDIA.
Fue introducida en Noviembre de 2006, y se basa en la utilización de un elevado número de
nodos de procesamiento para realizar operaciones sobre un gran volumen de datos en paralelo, consiguiendo reducir el tiempo de procesado y obteniendo altas prestaciones\cite{QECUDA}.

Uno de los principales objetivos de esta tecnologı́a es resolver problemas complejos que llevan asociados una alta carga computacional sobre la CPU de la forma más eficiente, haciendo uso de las GPUs incorporadas en la tarjeta gráfica. Para ello se realiza un procesado masivo de los datos, consiguiendo reducir considerablemente los tiempos de ejecución de la aplicación.

Antes de comenzar a paralelizar algún algoritmo o aplicación, es necesario llevar a cabo un estudio específico con el fin de identificar el mejor enfoque para el uso de los recursos ofrecidos por CUDA en cada caso. En la terminología de CUDA, la función o el código de programa que se ejecuta en la GPU reciben el nombre de \textit{kernel}. Un \textit{kernel} se procesa en paralelo por un conjunto de hilos que ejecutará las instrucciones de ese \textit{kernel} en una parte diferente de los datos en memoria. Los \textit{kernel} se lanzan en \textit{grids} y sólo un \textit{kernel} se ejecuta en un momento dado en una GPU. La GPU tiene varios multiprocesadores, por lo que los hilos son agrupados en bloques. La ejecución de bloque se lleva a cabo en los multiprocesadores. Al conjunto de bloques se le denomina \textit{grid}.

\begin{figure}
    \begin{center}
        \includegraphics[width=1\linewidth]{images/difference_CPU_GPU.png}
    \end{center}
    \caption{Diferencia de arquitecturas entre CPU (izquierda) y GPU(derecha). Extraído de \cite{kirk2016programming}}\label{}
\end{figure}

Hay varios tipos de memoria disponible en CUDA. Para  este proyecto se hará uso de la memoria global, la memoria constante y la memoria compartida. Cada una de estas memorias tiene diferente tipo de acceso y capacidad de almacenamiento.

La memoria global es la más utilizada en una GPU por su tamaño, aunque es de alta latencia pero mayor capacidad. Este tipo de memoria puede ser accedida desde cualquier multiprocesador durante el periodo de vida del programa. 

La memoria compartida es más rápida y de menor latencia que la memoria global. Cada multiprocesador tiene un límite total de memoria compartida que es particionada entre todos los hilos de los bloques. Cada bloque tiene su propio espacio de memoria compartida, esto significa que cada hilo sólo puede acceder a su propio espacio de memoria compartida. Esta memoria es declarada dentro de la función kernel pero compartiendo el mismo tiempo de vida con los hilos del bloque. 

La memoria constante, como su propio nombre indica, se usa para albergar datos que no cambian durante el transcurso de ejecución de un kernel. La capacidad de memoria constante disponible depende de la capacidad de cómputo de la GPU, y su principal ventaja es que en algunas situaciones el uso de memoria constante en lugar de la memoria global pueda reducir considerablemente el ancho de banda de memoria requerido por la aplicación. Dado que la memoria constante no puede ser cambiada en tiempo de ejecución, no es posible usarla para guardar los datos de las imágenes, por lo que en este caso particular, esta memoria sólo será usada para almacenar las máscaras de convolución las cuales son vectores de sólo lectura.

La utilización de una memoria u otra, va a depender de la aplicación que se esté desarrollando y de las características de la GPU con la que se cuente, pues dependiendo de la capacidad de la misma, varía el tamaño de memorias. 


\subsection{Convolución y Filtro de Sobel}

Antes de explicar en qué consiste el filtro de Sobel es importante mencionar que para que el filtro funcione, la imagen sobre la cual éste opera debe estar en escala de grises. Para esto se hace un procesamiento sobre la imagen original. Este proceso consiste en recorrer cada píxel de la imagen a color y modificar los canales RGB (Red, Green, Blue) con unos valores ya determinados\cite{gray} que juntos forman y dan color a un píxel. 

La convolución es una operación matemática que consiste en pasar un vector, llamado máscara de convolución, por cada uno de los pixeles vecinos de una imágen, multiplicar dichos pixeles y obtener su suma que serán almacenadas en el píxel donde nos encontramos.  Los operadores de Sobel combinan el suavizado y la diferenciación de Gauss, por lo que el resultado es más o menos resistente al ruido. Los valores de las máscaras están dados por dichos operadores, con los cuales se calculan los cambios horizontales y verticales en la imagen\cite{sobel}.


\begin{figure}[ht]
\centering
    \subfloat[Máscara de convolución.]{\includegraphics[width=1.0in]{images/mask.png}}
    \hspace{0.1cm}
    \subfloat[Imagen de entrada, se aplica la máscara sobre el píxel del centro y se suma la multiplicación de los pesos con sus vecinos (región azul).]{\includegraphics[width=1.0in]{images/convolucion.png}}
    \hspace{0.1cm}
    \subfloat[Salida del píxel del centro.]{\includegraphics[width=1.0in]{images/convolucion2.png}}
\caption{Elaboración propia}
\end{figure} 


Una vez se hayan detectado los cambios en X y Y con el recorrido de las máscaras de convolución, se haya la norma y la dirección del gradiente, el cual está expresado de la siguiente forma:

\begin{equation}
G = \sqrt[2]{Gx^2+Gy^2}
\end{equation}

\begin{equation}
\theta = atan\left(\frac{Gy}{Gx}\right)
\end{equation}

\cite{SD_openCV}\cite{sobelED}


\section{Implementación y resultados experimentales}

Se realizará una comparación de los tiempos obtenidos por cada implementación en los diferentes tipos de memoria de la GPU, y una implementación extra el cual mostrará los resultados de dicho algoritmo en la CPU (secuencial) por medio de la librería OpenCV.

En la tabla \ref{tab:caracteristicas}. se realiza una descripción de las características de cada una de las imágenes.


\begin{table}[ht]
\centering
\caption{Características de las imágenes. Elaboración propia.}
\label{tab:caracteristicas}
\begin{tabular}{lllll}

Imagen & Tamaño Imagen (Bytes) & Píxeles      &  &  \\
1      & 46398                 & 550 x 340    &  &  \\
2      & 116519                & 580 x 580    &  &  \\
3      & 86406                 & 638 x 640    &  &  \\
4      & 391282                & 1366 x 768   &  &  \\
5      & 616630                & 2560 x 1600  &  &  \\
6      & 2171320               & 4928 x 3264  &  &  \\
7      & 10178230              & 5226 x 4222  &  &  \\
8      & 2335593               & 12000 x 6000 &  &  \\
9      & 6977173               & 12000 x 9000 &  &  \\
10     & 5950639               & 19843 x 8504 &  & 
\end{tabular}
\end{table}


\subsection{Implementación secuencial}

OpenCV ofrece gran cantidad de funciones para procesamiento de imágenes, entre ellas se encuentra el filtro de Sobel. Para hacer uso de tal funcionalidad, es necesario proveerle una imagen de entrada, junto a otros parámetros, para ejecutar el filtro de Sobel. OpenCV usa internamente los hilos de la CPU para realizar estos procesos, es decir, la librería optimiza las funciones que implementa. Aún así, se puede lograr un mejor desempeño usando la GPU con CUDA, dado a las diferencias de la arquitectura ya antes mostrada entre la CPU y la GPU.

En la tabla \ref{tab:secuencial}  se muestran los tiempos de ejecución obtenidos al realizar el filtro de sobel con la librería Opencv para cada imagen. 

\begin{table}[ht]
\centering
\caption{Tiempos de ejecución con OpenCV. 
}
\label{tab:secuencial}
\begin{tabular}{lllll}
Imagen & Tiempo Promedio (segundos) &  &  &  \\
1      & 0,15288965                        &  &  &  \\
2      & 0,16600140                        &  &  &  \\
3      & 0,16263135                        &  &  &  \\
4      & 0,22078000                        &  &  &  \\
5      & 0,29979610                        &  &  &  \\
6      & 0,77888230                        &  &  &  \\
7      & 0,53865560                        &  &  &  \\
8      & 1,68595235                        &  &  &  \\
9      & 2,60327430                        &  &  &  \\
10     & 3,82147200                        &  &  & 
\end{tabular}
\end{table}

\begin{figure}[ht]
    \begin{center}
        \includegraphics[width=1.2\linewidth]{images/opencv.png}
    \end{center}
    \caption{Elaboración propia}
\end{figure}


\subsection{Implementación con memoria global}

Esta versión utiliza la memoria global (DRAM) de la GPU para almacenar, cargar y guardar el resultado del procesamiento en las imágenes. Esta memoria es de latencia alta pero es la de mayor capacidad, por esto es útil para almacenar los datos completos de las imágenes y máscaras. Esta memoria tiene un canal directo de comunicación con el host (CPU) por medio del puerto PCIExpress.

La figura \ref{fig:cats} muestra la imagen de entrada al algoritmo, la conversión a escala de grises y la imagen resultado con el filtro de Sobel.

\begin{figure}[ht]
\centering
    \subfloat[Imagen de entrada]{\includegraphics[width=2.0in]{images/cats.jpg}}
    \hspace{0.1cm}
    \subfloat[Imagen en escala de grises]{\includegraphics[width=2.0in]{images/Gray_image_CUDA.jpg}}
    \hspace{0.1cm}
    \subfloat[Image con el filtro e Sobel]{\includegraphics[width=2.0in]{images/Sobel_Global.jpg}}
\caption{Elaboración propia}\label{fig:cats}
\end{figure}


En la tabla \ref{tab:global}  se muestran los tiempos de ejecución obtenidos al realizar el filtro de Sobel con la memoria global para cada imagen. 

\begin{table}[ht]
\centering
\caption{Algoritmo con memoria global.}
\label{tab:global}
\begin{tabular}{lllll}
Imagen & Memoria Global (segundos) &  &  &  \\
1      & 0,00031665                &  &  &  \\
2      & 0,00055765                &  &  &  \\
3      & 0,00066810                &  &  &  \\
4      & 0,00165485                &  &  &  \\
5      & 0,00562505                &  &  &  \\
6      & 0,02804830                &  &  &  \\
7      & 0,02051475                &  &  &  \\
8      & 0,08429345                &  &  &  \\
9      & 0,12623075                &  &  &  \\
10     & 0,19830240                &  &  & 
\end{tabular}
\end{table}

\begin{figure}[ht]
    \begin{center}
        \includegraphics[width=1.2\linewidth]{images/global.png}
    \end{center}
    \caption{Elaboración propia}
\end{figure}


\subsection{Memoria Compartida}

La memoria compartida es otro tipo de memoria existente en la GPU (cache L1), con la diferencia de la global en que es accesada mucho más rápido por los hilos de ejecución y es compartida por un conjunto de hilos de cada bloque, también es más pequeña que la memoria global. Se hace necesario el uso de memoria compartida puesto que si todos los hilos de la GPU están tratando de acceder a la memoria global, se creará un cuello de botella, haciendo que nuestros programas se ejecuten más lento. 
Al hacer uso de la memoria compartida se aumenta la complejidad de entendimiento e implementación del algoritmo, puesto que se deben tener en cuenta muchas más variables, como de que se deben estar sincronizando los hilos de cada bloque para que no accedan a datos que todavía no han sido modificados por los otros hilos y el mapeo que debe hacerse a la memoria compartida desde la memoria global.

La figura \ref{fig:neumann} muestra la comparación entre la arquitectura de la GPU con una arquitectura moderna basada en el modelo propuesto por \textit{John Von Neumann}; se puede apreciar que la memoria compartida es de acceso mucho más rápido ya que se encuentra dentro del chip del procesador, mientras que acceder a la memoria global (DRAM) es mucho más "lejana" y por lo tanto más lenta al tratar de acceder a ella.

\begin{figure}[ht]
\centering
    \subfloat[Arquitectura de la GPU.]{\includegraphics[width=2.5in]{images/global_host.png}}
    \hspace{0.1cm}
    \subfloat[Arquitectura de una CPU moderna basada en el modelo de John Von Neumann. ]{\includegraphics[width=2.5in]{images/von_neumann_processor.png}}
\caption{Extraída de \cite{kirk2016programming}}\label{fig:neumann}
\end{figure}

En la tabla \ref{tab:compartida} se muestran los tiempos de ejecución obtenidos al realizar el filtro de Sobel con la memoria compartida para cada imagen.


\begin{table}[ht]
\centering
\caption{Algoritmo con memoria compartida}
\label{tab:compartida}
\begin{tabular}{lllll}
Imagen & Memoria Compartida (segundos) &  &  &  \\
1      & 0.00024280                    &  &  &  \\
2      & 0.00040695                    &  &  &  \\
3      & 0.00047495                    &  &  &  \\
4      & 0.00111000                    &  &  &  \\
5      & 0.00400425                    &  &  &  \\
6      & 0.01429470                    &  &  &  \\
7      & 0.01944510                    &  &  &  \\
8      & 0.06232435                    &  &  &  \\
9      & 0.08929670                    &  &  &  \\
10     & 0.13908525                    &  &  & 
\end{tabular}
\end{table}

\begin{figure}[ht]
    \begin{center}
        \includegraphics[width=1.0\linewidth]{images/compartida.png}
    \end{center}
    \caption{Elaboración propia}
\end{figure}


\subsection{Memoria constante}

CUDA pone a nuestra disposición otro tipo de memoria conocida como memoria constante. Como su nombre lo indica, utilizamos la memoria constante para datos que no cambiarán en el transcurso de la ejecución del kernel. NVIDIA proporciona 64 KB de memoria constante que trata de manera diferente a como trata la memoria global estándar. En algunas situaciones, usar memoria constante en lugar de memoria global reducirá el ancho de banda de memoria requerido. En esta memoria sólo se almacenará la máscara de convolución puesto que no permite escritura en tiempo de ejecución\cite{constantMemory}.

La figura \ref{fig:constantDistrib} muestra cómo está distribuida la memoria constante en la GPU.

\begin{figure}[ht]
    \begin{center}
        \includegraphics[width=1.0\linewidth]{images/constDistrib.png}
    \end{center}
    \caption{Extraído de \cite{kirk2016programming}}\label{fig:constantDistrib}
\end{figure}


En la tabla \ref{tab:const} se muestran los tiempos de ejecución obtenidos al realizar el filtro de Sobel con la memoria constante para cada imagen. 

\begin{table}[ht]
\centering
\caption{Algoritmo con memoria constante}
\label{tab:const}
\begin{tabular}{lllll}
Imagen & Memoria Constante (segundos) &  &  &  \\
1      & 0,00028225                   &  &  &  \\
2      & 0,00045420                   &  &  &  \\
3      & 0,00052085                   &  &  &  \\
4      & 0,00108675                   &  &  &  \\
5      & 0,00347660                   &  &  &  \\
6      & 0,01665020                   &  &  &  \\
7      & 0,01216705                   &  &  &  \\
8      & 0,05257680                   &  &  &  \\
9      & 0,11910089                   &  &  &  \\
10     & 0,11962325                   &  &  & 
\end{tabular}
\end{table}

\begin{figure}[ht]
    \begin{center}
        \includegraphics[width=1.2\linewidth]{images/constante.png}
    \end{center}
    \caption{Elaboración propia}
\end{figure}


\section{Comparación de resultados}

En la figura \ref{fig:comp} se encuentran las comparaciones en tiempos de cada uno de los algoritmos implementados.

\begin{figure}[ht]
\centering
    \subfloat[Comparación de tiempos entre memorias de la GPU y algoritmo secuencial]{\includegraphics[width=3.5in]{images/comparacion_memorias.png}}
    \hspace{0.1cm}
    \subfloat[Comparación de tiempos entre memorias de la GPU]{\includegraphics[width=3.5in]{images/comparacion_memorias_GPU.png}}
\caption{Elaboración propia}\label{fig:comp}
\end{figure}

\section{Conclusiones}

\begin{itemize}
\item El uso del paralelismo (en las tres implementaciones con los diferentes tipos de memorias) muestra una evidente mejora en el tiempo de procesamiento sobre cada una de las imágenes en comparación con la versión secuencial (implementación ofrecida por OpenCV)
\item La mejora de procesamiento al usar paralelismo con GPU y CUDA se hace más evidente cuando la cantidad de datos a procesar es muy grande. En este caso se hace evidente con las imágenes de mayor cantidad de píxeles, como las imágenes 8(12000x6000px), 9(12000x9000px) y 10(19843x8504px)
\item Se logra evidenciar como  el uso de memorias de baja latencia como la constante y la compartida permite una mejora en el rendimiento de la implementación. 
\item En la implementación de memoria compartida se usó una combinación entre esta memoria, para guardar temporalmente los datos de la imagen, y la memoria constante para almacenar las máscaras de convolución. Se esperaba que esta implementación fuera la más rápida. La razón por la que no sucede esto es porque la cantidad de operaciones que realiza cada hilo de la GPU accediendo a los datos de la memoria compartida es muy poca, dicho de otra forma, cada hilo sólo ejecutará la cantidad de operaciones dada por el ancho máximo de la máscara. Son muy pocas operaciones en las cuales no se aprovecha al máximo esta técnica si se compara con un producto punto.
\end{itemize}

\section{Trabajos Futuros}

\begin{itemize}
\item Mejorar el desempeño con el uso de warps aprovechando también mayor capacidad de procesamiento de la GPU
\item Hacer uso de la memoria compartida junto con la memoria constante para compararlo con los resultados ya obtenidos en este informe.
\end{itemize}

% if have a single appendix:
%\appendix[Proof of the Zonklar Equations]
% or
%\appendix  % for no appendix heading
% do not use \section anymore after \appendix, only \section*
% is possibly needed

% use appendices with more than one appendix
% then use \section to start each appendix
% you must declare a \section before using any
% \subsection or using \label (\appendices by itself
% starts a section numbered zero.)
%


% \appendices
% \section{Proof of the First Zonklar Equation}
% Appendix one text goes here.

% % you can choose not to have a title for an appendix
% % if you want by leaving the argument blank
% \section{}
% Appendix two text goes here.


% use section* for acknowledgment
% \ifCLASSOPTIONcompsoc
%   % The Computer Society usually uses the plural form
%   \section*{Acknowledgments}
% \else
%   % regular IEEE prefers the singular form
%   \section*{Acknowledgment}
% \fi


% The authors would like to thank...


% Can use something like this to put references on a page
% by themselves when using endfloat and the captionsoff option.
\ifCLASSOPTIONcaptionsoff
  \newpage
\fi



% trigger a \newpage just before the given reference
% number - used to balance the columns on the last page
% adjust value as needed - may need to be readjusted if
% the document is modified later
%\IEEEtriggeratref{8}
% The "triggered" command can be changed if desired:
%\IEEEtriggercmd{\enlargethispage{-5in}}

% references section

% can use a bibliography generated by BibTeX as a .bbl file
% BibTeX documentation can be easily obtained at:
% http://mirror.ctan.org/biblio/bibtex/contrib/doc/
% The IEEEtran BibTeX style support page is at:
% http://www.michaelshell.org/tex/ieeetran/bibtex/
%\bibliographystyle{IEEEtran}
% argument is your BibTeX string definitions and bibliography database(s)
%\bibliography{IEEEabrv,../bib/paper}
%
% <OR> manually copy in the resultant .bbl file
% set second argument of \begin to the number of references
% (used to reserve space for the reference number labels box)
% \begin{thebibliography}{1}

% % \bibitem{IEEEhowto:kopka}
% % H.~Kopka and P.~W. Daly, \emph{A Guide to {\LaTeX}}, 3rd~ed.\hskip 1em plus
% %   0.5em minus 0.4em\relax Harlow, England: Addison-Wesley, 1999.

\renewcommand\refname{Referencias}

% \end{thebibliography}
\bibliographystyle{IEEEtran}
\bibliography{IEEEabrv,bibfile}


% biography section
% 
% If you have an EPS/PDF photo (graphicx package needed) extra braces are
% needed around the contents of the optional argument to biography to prevent
% the LaTeX parser from getting confused when it sees the complicated
% \includegraphics command within an optional argument. (You could create
% your own custom macro containing the \includegraphics command to make things
% simpler here.)
%\begin{IEEEbiography}[{\includegraphics[width=1in,height=1.25in,clip,keepaspectratio]{mshell}}]{Michael Shell}
% or if you just want to reserve a space for a photo:

% \begin{IEEEbiography}{Michael Shell}
% Biography text here.
% \end{IEEEbiography}

% % if you will not have a photo at all:
% \begin{IEEEbiographynophoto}{John Doe}
% Biography text here.
% \end{IEEEbiographynophoto}

% insert where needed to balance the two columns on the last page with
% biographies
%\newpage

% \begin{IEEEbiographynophoto}{Jane Doe}
% Biography text here.
% \end{IEEEbiographynophoto}

% You can push biographies down or up by placing
% a \vfill before or after them. The appropriate
% use of \vfill depends on what kind of text is
% on the last page and whether or not the columns
% are being equalized.

%\vfill

% Can be used to pull up biographies so that the bottom of the last one
% is flush with the other column.
%\enlargethispage{-5in}



% that's all folks
\end{document}


